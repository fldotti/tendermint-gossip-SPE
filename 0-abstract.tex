%\fd{Title is provisory}
%
%\fd{Color legend: {\new text blue is new}. }
%
%\fd{{\old black is non modified from 'Paxos Gossip - maybe unuseful but here to check it and remember discussions.}}
%
%\fd{\color{violet} violet is existing text reviewed/modified to this paper}
%
%\fd{{\color{teal}parts in in green/teal is old text that is valid for this paper}.  
%   we have to see how to reuse or not.   this is to be considered a new paper, not an extension ... correct ?}
%
%\fd{Draft abstract to check if these are the main aspects.}
%\new
State Machine Replication has been extensively used to provide highly-available, strongly-consistent systems.  Modern systems often require additionally to scale to several nodes.  This is the case of 
%This is the case of several existing state machine replication based systems that have to span service provisioning to a wider geographical scope. 
several blockchains, that have to run consensus among dozens to hundreds of nodes.

Gossip allows to scale communication to larger sets of nodes and has been recently considered as communication layer to support consensus.
%
By their nature, both consensus protocols and gossip handle message losses and process failures.   This double redundancy means that it is not yet clear how efficient is to use gossip as a black building-box for consensus.

In this paper, we investigate these aspects in the context of blockchains.
%Due to their stringent scalability needs, this paper investigates    
%
The paper discusses how far redundancy can be eliminated without compromising consensus' safety and liveness.
%
Then it proposes to selectively eliminate redundancy considering the combined use of protocols.
%
To accomplish that, a cross layer mechanism is proposed that, while keeping modularity, uses message semantics from consensus at the gossip level to reduce overhead while forwarding messages.   With a prototype implementing the ideas, experiments with 32 and 128 nodes show that the mechanisms proposed respectively enhance throughput 1.24 $\times$ and  $3.42 \times$ while slightly reducing latency.  
% this is derived from pictures 3 to 6

Furthermore, we argue and show results that the proposed redundancy elimination mechanisms do not harm resiliency...

%\old


% Gossip-based consensus protocols have been recently proposed to confront the challenges faced by state machine replication in large geographically distributed systems.

% It is unclear, however, to which extent consensus and gossip communication fit together.

% On the one hand, gossip communication has been shown to scale to large settings and efficiently handle participant failures and message losses. 

% On the other hand, gossip may slow down consensus. 

% Moreover, gossip's inherent redundancy may be unnecessary since consensus naturally accounts for participant failures and message losses. 

% This paper investigates the suitability of gossip as a communication building block for consensus.

% We answer three questions: How much overhead does classic gossip introduce in consensus? Can we design consensus-friendly gossip protocols? Would more efficient gossip protocols still maintain the same reliability properties of classic gossip?
