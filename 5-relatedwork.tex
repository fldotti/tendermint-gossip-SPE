%!TEX root =  report.tex
\section{Related work}
\label{sec:relwork}
\color{teal}
Gossip algorithms were first introduced by Demers et al.~\cite{demers87} to
manage replica consistency in the Xerox Clearinghouse Service~\cite{oppen83}.
The proposed algorithms were specific for the dissemination of database
updates, assumed to not be very frequent (a few per second, at most).
%
The adoption of gossip mechanisms as a building block for the dissemination of
arbitrary application messages derives from Bimodal Multicast~\cite{Birman99}.
The algorithm consists of two phases.
In the first phase, messages are disseminated in a best-effort fashion through
multicast trees, using IP-multicast when available.
In the second phase, processes periodically send to a random-selected peer a list
of recently received messages, so that to retransmit, on demand, messages that
have not yet been received by the peer.
%
Since then, multiple approaches have been proposed to improve throughput and
robustness of gossip dissemination~\cite{birman01, eugster03, gupta02, kempe04,
lin99, leitao07, melamed04, vogels03}.

Research in gossip-based broadcast algorithms has focused essentially on two
issues.
%
First, the efficient dissemination of messages in large-scale systems through
the adoption of overlay networks.
Proposed approaches consider building pseudo-random network overlays,
by selecting links based on geographic proximity and available
bandwidth~\cite{kempe04, melamed04}, or topological and connectivity
properties~\cite{lin99, leitao07, voulgaris13}.
%
A second research direction addresses the cost/effectiveness of epidemic
mechanisms which enable processes to request messages that they failed to
receive.
The efficiency of these anti-entropy~\cite{demers87, Birman99} or gossip
repair~\cite{birman01, eugster03, gupta02, vogels03} mechanisms
is crucial to improve the reliability of gossip dissemination.
%
Efforts have also been made to develop gossip-based services to support
large-scale broadcast and multicast algorithms, such as failure
detection~\cite{renesse98}, group membership~\cite{ganesh03, johansen06},
monitoring and management systems~\cite{renesse02}.

Semantic Gossip differs from existing approaches because it is designed to
support distributed applications that, by themselves, include layers of
redundancy.

\color{black}

This is the case of Paxos, which includes both typical broadcast steps (to
propose values) and the exchange of control messages to ensure agreement, which
is a strong form of reliability.

Probabilistic Atomic Broadcast~\cite{felber02} is the algorithm whose behavior
most resembles the operation of Paxos atop gossip.
%is most similar to 
The algorithm proceeds in rounds, in each round a process can broadcast a
message and should vote for a message, either broadcast or received during the
round.
Processes periodically exchange the list of messages and associated votes with
a random subset of peers.
When the number of votes reaches a threshold, all messages in the list are
delivered, and the process proceeds to the next round.
%
As in our Paxos deployment, processes send and forward values (broadcast
messages) and votes to peers via gossip.
Unlike Paxos, the algorithm of~\cite{felber02} only provides probabilistic
safety guarantees: two processes may deliver messages in distinct orders, which
is equivalent in Paxos to deciding different values in the same consensus
instance.

\color{teal}

Even though most work on gossip has considered benign failures (e.g., process crashes), 
recent Byzantine fault tolerant consensus protocols for large-scale environments (e.g., blockchain) 
have considered the use of gossip as underlying communication substrate.
Tendermint
%\footnote{\url{https://tendermint.com/}} 
is a blockchain middleware
based on a BFT consensus algorithm~\cite{bucham18} designed for gossip
communication.  Tendermint has its own gossip layer implementation, that is
application-specific and tightly coupled with the consensus implementation.
%
Casper~\cite{buterin17}, the BFT consensus algorithm proposed to replace the
proof-of-work core of the Ethereum blockchain is also designed for a
gossip-based environment.
%\footnote{\url{https://ethereum.org/}}
%
HotStuff~\cite{maofan18}, the BFT consensus protocol at the core of the Libra
Blockchain~\cite{amsden20}, although not designed for gossip-based
communication, considers its adoption as the number of processes participating
on consensus (validator nodes) grows~\cite{libra18}.
The key architectural aspect that distinguishes these proposals from Semantic
Gossip is that gossip in blockchain systems is intertwined with consensus
logic.  Semantic Gossip exploits application (i.e., consensus) semantics
without giving up modularity.

\color{black}